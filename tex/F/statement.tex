There are $n$ children (numbered from $1$ to $n$) learning
the arithmetic operations, which include \emph{addition} ``$+$'', 
\emph{subtraction} ``$-$'', 
\emph{multiplication} ``$\times$'', and \emph{division} ``$\div$'' on 
rational numbers.

In the beginning, each child has a paper sheet with only a zero on it. 
Their teacher, Frank, will then give them $q$ operations. 
The $i$-th operation consists of an operator $c_i$ and an integer $x_i$.
The children numbered $\ell_i,\ell_i+1,\dots,r_i$ have to append the 
operator $c_i$ and the integer $x_i$ to their paper sheets.
After that, every child has an expression on their sheet to be evaluated.

For example, suppose that $n=3$, $q=2$, $c_1$ is ``$+$'', $x_1=1$, $\ell_1=1$, 
$r_1=2$, $c_2$ is ``$-$'',  $x_2=2$, $\ell_2=2$, and $r_2=3$. 
The expressions on the sheets are are $0+1$, $0+1-2$ and $0-2$ for 
children 1, 2 and 3, respectively.

Since Frank is really lazy and wants to verify the answers quickly,
he asks you to calculate the sums of the values of all children's 
expressions. 
Suppose that the value of the expression assigned to child $i$ is 
$\frac{a_i}{b_i}$, 
then the value will be $a\times b^{-1}\bmod 10^9+7$ instead, 
where $b^{-1}$ denotes the integer satisfying 
$b\times b^{-1}\equiv 1\mod 10^9+7$. 
If the sum is not in $[0, 10^9+7)$, then the sum modulo $10^9+7$ should be
returned to Frank.

Note: The arithmetic operations has PEMDAS rule, that is, 
multiplications and divisions should be evaluated before evaluating additions
and subtraction.
